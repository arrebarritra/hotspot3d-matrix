The repo containing our implementation can be found on GitHub:
\href{https://github.com/arrebarritra/hotspot3d-matrix}{hotspot3d-matrix}

\hypertarget{group-member-contributions}{%
\subsection{Group Member
Contributions}\label{group-member-contributions}}

\hypertarget{aritra-bhakat}{%
\subsubsection{Aritra Bhakat}\label{aritra-bhakat}}

\hypertarget{ludwig-kristoffersson}{%
\subsubsection{Ludwig Kristoffersson}\label{ludwig-kristoffersson}}

\hypertarget{saga-palmuxe9r}{%
\subsubsection{Saga Palmér}\label{saga-palmuxe9r}}

\hypertarget{introduction}{%
\section{Introduction}\label{introduction}}

Hotspot3d tries to solve the heat equation, given by the following PDE:

\begin{align}
    \frac{\partial T}{\partial t} = \alpha \nabla^2 T(\mathbf{x}) + \beta P(\mathbf{x}) + \alpha \left[ T_{\text{amb}} - T(\mathbf{x}) \right] \\
    = \alpha \left( \frac{\partial T}{\partial x} + \frac{\partial T}{\partial y} + \frac{\partial T}{\partial z} \right) + \beta P(\mathbf{x}) + \alpha \left[ T_{\text{amb}} - T(\mathbf{x}) \right]
\end{align}

where \(T\) is the temperature field, \(P\) is the power field,
\(T_{\text{amb}}\) is the ambient temperature, and
\(\mathbf{\alpha}, \beta\) are arbitrary constants which have to do with
chip properties.

The PDE is discretised using forward difference for the time step and
second order central differences for the Laplace operator:

\begin{align}
    T^{t+1}_{ijk} =\\
    T^t_{ijk}
    + \left[ c_w T^t_{i-1,jk} - (c_w + c_e) T^t_{ijk} + c_e T^t_{i+1,jk} \right]\\
    + \left[ c_t T^t_{i,j-1,k} - (c_t + c_s) T^t_{ijk} + c_s T^t_{i,j+1,k} \right]\\
    + \left[ c_b T^t_{ij,k-1} - (c_b + c_t) T^t_{ijk} + c_t T^t_{ij,k+1} \right]\\
    + \text{sdc} \cdot P_{ijk} \\
    + c_t [T^t_{\text{amb}} - T^t_{ijk}] \\
    = c_c T^t_{ijk} \\
    + c_w T^t_{i-1,jk} + c_e T^t_{i+1,jk} \\
    + c_t T^t_{i,j-1,k} + c_s T^t_{i,j+1,k} \\
    + c_b T^t_{ij,k-1} + c_t T^t_{ij,k+1} \\
    + \text{sdc} \cdot P_{ijk} + c_t T^t_{\text{amb}}
\end{align}

Here, the constants \(c_x\) are combinations of the constant
\(\alpha_x\), along with the timestep \(dt\), size of the chip in the
given dimension and amount rows/cols/layers for the given dimension. The
letters \(w,e,n,s,b,t\) stand for the compass dimensions and bottom and
top, each pair of irections corresponding the \(xyz\) axes. The values
are symmetrical, meaning \(c_w = c_e \ldots\) and so on. To simplify the
computation, the \(T^t_{ijk}\) terms are gathered into the constant
\(c_c = 1 - (2 \cdot c_w + 2 \cdot c_n + 3 \cdot c_t)\). The constant
\(\text{sdc}\) equals \(dt / \text{capacitance}\). The temperature \(T\)
and the power \(P\) both have dimensions \(n_x \times n_y \times n_z\).

\hypertarget{methodology}{%
\section{Methodology}\label{methodology}}

We observed that the discretised PDE corresponds well to sparse matrix
multiplication. Our approach is to use a sparse matrix for each
dimension to calculate the second order central difference part of the
discretised equation. The rest of the equation can be calculated as
trivial (vector) addition.

Our data has the \(n_x \times n_y \times n_z\). Our differentiation
matrices are sparse square matrices with dimensions \(n_x\), \(n_y\) and
\(n_z\) respectively:

\begin{align}
    \partial X =
    \begin{bmatrix}
    c_w + c_c   &c_e    &&&&&&0 \\
    c_w     &c_c   &c_e \\
    &c_w    &c_c    &c_e \\
    &&&\ldots \\
    &&&&&c_w    &c_c   &c_e \\
    0&&&&&&c_w   &c_c + c_e
    \end{bmatrix}
\end{align}

\begin{align}
    \partial Y =
    \begin{bmatrix}
    c_n   &c_s    &&&&&&0 \\
    c_n     &0   &c_s \\
    &c_n    &0    &c_s \\
    &&&\ldots \\
    &&&&&c_n    &0   &c_s \\
    0&&&&&&c_n   &c_s
    \end{bmatrix}
\end{align}

\begin{align}
    \partial Y =
    \begin{bmatrix}
    c_b   &c_t    &&&&&&0 \\
    c_b     &0   &c_t \\
    &c_b    &0    &c_t \\
    &&&\ldots \\
    &&&&&c_b    &0   &c_t \\
    0&&&&&&c_b   &c_t
    \end{bmatrix}
\end{align}

Note that \(\partial Y\) and \(\partial Z\) are missing \(c_c\) in the
diagonal as the term is already accounted for in \(\partial X\). These
matrices operate on the 3D temperature data.

Since we can only work with matrices and not directly with 3D tensors
within the cuSPARSE library, we need to configure the matrix layouts for
the different computations to give us the correct result.

We apply \(\partial X\) and \(\partial Y\) on \(n_z\) batches of
\(n_x \times n_y\) matrices.

\begin{align}
    T_{XY} =
    \begin{bmatrix}
    T_{111} &\ldots &T_{1,n_x,1} \\
    &\ldots \\
    T_{n_y,11} &\ldots & T_{n_y,n_x,1}
    \end{bmatrix}
    \quad
    \ldots
    \quad
    \begin{bmatrix}
    T_{11,n_z} &\ldots &T_{1,n_x,n_z} \\
    &\ldots \\
    T_{n_y,1,n_z} &\ldots & T_{n_y,n_x,n_z}
    \end{bmatrix}
\end{align}

We call this layout \(T_{XY}\).

When we apply \(\partial X\), we need to take the transpose of
\(T_{XY}\) so that the \(x\)-dimension is lined up with the column. We
then need to transpose the result so the \(x\) dimension is lined up
with the row again. We perform the following calculation:

\[
\left( \partial X \cdot T^{\intercal}_{XY} \right)^{\intercal} =  T_{XY} \cdot\partial X^{\intercal}
\]

For \(\partial Y\) the calculation is straightforward:

\[
\partial Y \cdot T_{XY}
\]

For \(\partial Z\) we need to change the layout of the \(T\). We flatten
the data into one big \(n_z \times (n_x \cdot n_y)\) matrix, so that
values adjacent on the \(z\)-axis are lined up in one column:

\begin{align}
    \begin{bmatrix}
    T_{111} &\ldots &T_{n_y,n_x,1} \\
    T_{112} &\ldots &T_{n_y,n_x,2} \\
    &\ldots \\
    T_{11,n_z} &\ldots & T_{n_y,n_x,n_z}
    \end{bmatrix}
\end{align}

We call this layout \(T_Z\). The calculation we perform now is also
straightforward:

\[
\partial Z \cdot T_{Z}
\]

Putting this together, we can rewrite the discretised equation as:

\[
T^{t+1} = T^{t}_{XY} \cdot \partial X^{\intercal} + \partial Y \cdot T^{t}_{XY} + \partial Z \cdot T^t_Z + \text{sdc} \cdot P + c_t T_{\text{amb}} I
\]

\hypertarget{experimental-setup}{%
\section{Experimental Setup}\label{experimental-setup}}

\hypertarget{results}{%
\section{Results}\label{results}}

\hypertarget{discussion}{%
\section{Discussion}\label{discussion}}

\hypertarget{references}{%
\section{References}\label{references}}
